\documentclass{article}
\usepackage[utf8]{inputenc}

\title{COMP1204 Data management: Coursework 2}
\author{Kritagya Gurung}


\begin{document}

	
	\maketitle
	\begin{center}
		Student ID: 30220238
	\end{center}
	
	\newpage

	\section{The Relational Model}

	\subsection{EX1}
	
	\begin{center}
		\begin{tabular}{||c c | c c | c c ||} 
			\hline
			Attribute name & data type & Attribute name & data type & Attribute name & data type \\
			\hline
			Author & String & No. helpful & int & Location & int \\ 
			Content & String & Overall & int & Cleanliness & int \\ 
			Date & String & Value & int & Check in/Front desk & int \\ 
			No. readers & int & Rooms & int & Service & int \\
			Business service & int & - & - & - & -\\
			\hline
		
		\end{tabular}
	\end{center}
	Relation Schema: 
	\newline
	HotelReview(Author,Content,Date,No. reader,No. helpful,Overall,Value,Rooms,Location,Cleanliness,Check in/Front desk,Service,Business service).
	\newline
	Author attribute is the primary key since all other attributes rely on the author and in order to write a review, you must have an author (no fields in this attribute can be null).
	
	\subsection{EX2}
	\begin{math}
		\textrm{Author}
		\rightarrow
		\textrm{Content,Date,No. reader,No. helpful,Overall,Value,Rooms,Location,Cleanliness,Check in/Front desk,Service, Business service}
	\end{math}
	\newline
	Author (attribute) should be the determinant for all other attributes since it is the author who decides the scores and the content of the review on a hotel.
	\newline
	Candidate keys:
	\newline
	Author, Author + Content, Author + Content + Date. 
	\subsection{EX3}
	
	\section{Entity-Relationship Diagramming}
	
	\subsection{EX4}
	
	\section{Relational Algebra}
	
	\subsection{EX5}
	\begin{eqnarray}
	\sigma_{User_ID = given_ID}(Review)
	\end{eqnarray}
		
	\subsection{EX6}
	\begin{eqnarray}
	\sigma _{No. reviews given > 2}((User_ID) \gamma _{count(User_ID) \rightarrow No. reviews given}(Review)) \bowtie Author
	\end{eqnarray}
	
	\subsection{EX7}
	
	\subsection{EX8}
	
	\section{SQL Queries}
	
	\subsection{EX9}
	
	\subsection{EX10}
	
	\subsection{EX11}
		
	\subsection{EX12}
	
	\subsection{EX13}
	
	\subsection{EX14}
	
	\section{Conclusions}
	
	\subsection{EX15}

\end{document}